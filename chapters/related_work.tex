\chapter{Related work}
In this chapter I will attempt to review a number of projects similar to this one. They will be evaluated based on their perceived usefulness to this project, which is comprised of: licensing, embedded scripting language, networking support and code quality.

    	\section{Flare Isometric Engine}
    	Flare\cite{flare} is an open source isometric hack and slash game engine. It uses the SDL library for displaying graphics, and simple text based file formats. It is in active development at time of writing, and is available under the GPLV3.
    	It does not appear to have any embedded scripting language.
    	
    	\section{Holyspirit}
    	Holyspirit\cite{holyspirit} claims to be in alpha. It uses the SFML library. Does not appear to support networking.
    	Developed in French. Unlike the above, it is an actual game in the hack and slash genre, as opposed to a game engine (although an engine is, of course a part of it). It is also available under a permissive license.
    	
    	\section{Fifengine}
    	Fifengine\cite{fife} (Flexible Isometric Free Engine) is a FOSS generic isometric game engine.
    	It supports python scripting, and the UI is skinnable with xml.
    	It uses SDL and opengl. It does not support networking.
    	Like Flare, it is a game engine, not a game, but it is more generic, in that is is intended for all kinds of isometric games.
    	
    	\section{ProjectDDT}
    	ProjectDDT\cite{ddt} is an existing attempt to create a modern FOSS engine for Diablo.
    	It has been abandoned now since 2011.
    	Extending this project was considered over creating a new one, but this was decided against as the existing code appears unmaintainable and quite hard to follow.
    	It is however, very useful as a reference, as it is under the GPL.
    	It contains code for loading and interpreting several diablo file formats which proved useful.
    	
    	\section{Diablo 1 HD Mod}
    	This\cite{d1hd} is another recreation of the diablo engine that is far more advanced than ProjectDDT. The game appears to be fully playable. However, the source code is not available, and there does not appear to be any plans for it to be made available at any point.

\mbox{}

\noindent
Ultimately, I chose to make my own new engine, as I wanted to have the experience of designing a game engine from scratch.