\documentclass[a4paper]{article}

\usepackage{url}

\title{Isometric Hack and Slash Game Engine}
\author{Tom Mason - 10329877\\Supervisor: Mads Haahr}

\begin{document}
\maketitle

\begin{abstract}
The purpose of the project is to create a modern open source reimplementation of the engine used in the 1996 video game Diablo.\\
It will use the original data files, so as to avoid issues with copyright, but will also
support modern file formats, and be a generic engine for games of that style.\\
The original game is an isometric top down hack and slash game, which features some roguelike elements, such as
random items and dungeons. These will be a focus of the project.
\end{abstract}

\newpage

\tableofcontents

\newpage

\section{Background Research}
    \subsection{Relevant existing FOSS Isometric Engines}
    	\subsubsection{Flare Isometric Engine}
    	Flare\cite{flare} is an open source isometric hack and slash game engine. It uses the SDL library for displaying graphics, and simple text based file formats.
    	It does not appear to have any embedded scripting language.
    	
    	\subsubsection{Holyspirit}
    	Holyspirit\cite{holyspirit} claims to be in alpha. It uses the SFML library. Does not appear to support networking.
    	Developed in French.
    	
    	\subsubsection{Fifengine}
    	Fifengine\cite{fife} (Flexible Isometric Free Engine) is a FOSS generic isometric game engine.
    	It supports python scripting, and the UI is skinnable with xml.
    	It uses SDL and opengl. It does not support networking.
    	
    	\subsubsection{ProjectDDT}
    	ProjectDDT\cite{ddt} is an existing attempt to create a modern FOSS engone for Diablo.
    	It has been abandoned now since 2011.
    	I have decided to create my own project, in favour of continuing work on this, because it is extremely badly engineered.
    	For example: the creator of this work decided not to use header files, but simply placed function declarations at the start of files where he needed them.
    	The code is also just generally not very readable, and he has a tendency to not create classes where they would have been appropriate.
    	It is however, very useful as a reference, as it is under the GPL.
    	It contains code for loading and interpreting several diablo file formats, which while hard to follow, is still extremely useful.
    
    \subsection{Design Choices}
    The engine should support python scripting, to allow entension of the engine, and of games created forthe engine.
    File formats used by the game should be simple text formats, like the formats used in Fifengine.
    The engine should be divided into a number of module.
    
    \newpage
    
    \subsection{Libraries}
        \subsubsection{2d graphics libraries}
    	There seems to be 3 different options for 2d graphics in C++:
    	\begin{itemize}
    	    \item{SDL}
    	    \item{Allegro}
    	    \item{SFML}
    	\end{itemize}
    	
    	Of the above, all are written in plain C, except Allegro, which is C++.
    	I have decided to use SDL for this project, as I am already familiar with it.
    	More specifically, I have decided to use SDL 1.
    	SDL 2 has been released, but is not yet packaged in most distros. 
    	The intent is to write an SDL backend, which will eventually support either SDL1 or 2.
    	
    	\subsubsection{Cross Platform}
        The Boost C++ library addressess many of the problems with writing portable C++ code today.
        Specifically, I intend to make use of the boost::filesystem and boost::threads modules to provide platform-agnostic access to threads and files.
        Even with boost::filesystem, I shall have to take care to use case insensitive file loading, as the original game was written for windows, so filename cases may not be consistent.
        
        \subsubsection{Audio}
        SDL has a module for audio, SDL\_sound\cite{sdls}, but it has not been updated since 2008.
        FFMPEG's library, libavcodec\cite{libavcodec} supports a lrge number of formats.
        OpenAL seems to be popular also, but is no longer FOSS.
    
\newpage
    
\bibliographystyle{plain}
\bibliography{references}



\end{document}
